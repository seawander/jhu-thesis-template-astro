\chapter{Chapter with Citation in ApJ Style}\label{chapter-apj}
\paragraph*{Abstract} Put your abstract in.
\section{Citation Examples}
You can cite a reference using \textbackslash{citet\{\}}, \textbackslash{citep\{\}}, and \textbackslash{citealp\{\}}. They will look like this:

\citet{ren18}, 

\citep{ren18},

and \citealp{ren18}.

You can just copy the AAS bibtex entires from ADS\footnote{\url{https://ui.adsabs.harvard.edu}}, and paste them in the .bib file.

\section{Equation and Table Examples}\label{sec:alpha}
This is your section alpha. 
\subsection{Equation}\label{eq-sample}
This is an equation sample taken from \citet{ren18}:
\begin{equation}
T_{\text{NMF}} = D_{\text{NMF}} + S_{\text{NMF}},\label{eq-example}
\end{equation}
where the subscript ${}_{\text{NMF}}$ means performing the NMF modeling result for the stellar signal ($S$) or disk signal ($D$) alone.

You can refer to the equation as Equation~\eqref{eq-example}.

\subsection{Long Table}\label{appendixSymbols}
Table~\ref{tab:nmf-sym} is part of a long table that was previous presented in an ApJ paper \citep{ren18}. The format was changed from the deluxetable style in ApJ.

{
\fontsize{10}{12}
\begin{center}
\begin{longtable}{cp{2cm}cp{6cm}}
\caption[Short table description 1.]{Long description of table 1}\label{tab:nmf-sym}\\
\hline\hline
Symbol	&	Expression 	& Dimension	& Meaning \\
\hline
\endfirsthead
\multicolumn{4}{l}{\sl \tablename\ \thetable{} -- (continued)} \\
\hline\hline
Symbol	&	Expression 	& Dimension	& Meaning \\
\hline
\endhead
\hline
{\sl (continued)} \\
\endfoot
\hline

\endlastfoot
$\circ$ & $(A\circ B)_{ij} = A_{ij}B_{ij}$ & & Element-wise (Hadamard) multiplication for matrices $A$ and $B$ of same dimension.\\
$D$ & & $1\times N_{\text{pix}}$ & Flattened image of the astrophysical signal (i.e., no stellar information).\\
$\hat{D}$ & $T - \hat{f}T_{\text{NMF}}$& $1\times N_{\text{pix}}$ & Reduced best image of the astrophysical signal ($D$), obtained from BFF procedure.\\
$D_f$ & $T - fT_{\text{NMF}}$& $1\times N_{\text{pix}}$ & Reduced image of the astrophysical signal with scaling factor $f$.\\
$D_{\text{NMF}}$& $\omega^{({D})} H$& $1\times N_{\text{pix}}$ & NMF model of the astrophysical signal ($D$).\\
$\delta(\cdot)$ & &  & The change of the $(\cdot)$ item after one iteration.\\
$F_{\mathrm{disk}}/F_{\mathrm{star}}$ & & & Flux ratio between the disk and the star.\\
$f$ & &  & Scaling factor, where $0 < f < 1$.\\
$\hat{f}$ & &  & Optimum scaling factor obtained from the BFF procedure, corresponding with $\hat{D}$.\\
$H$, $H^{(k)}$, $H^{(k+1)}$ & $[H_1^{T}, \cdots, H_n^T]^T$ & $n \times N_{\text{pix}}$ & NMF component matrix for the reference cube.\\
$H_1$, $H_i$, $H_n$ &  & $ 1 \times N_{\text{pix}}$ & The $1$-st, $i$-th, and $n$-th NMF component for the reference cube ($R$).\\
${(\cdot)}^{(k)}$, ${(\cdot)}^{(k+1)}$ & superscript &  & Iteration step number.\\
$\mu_f^{(k)}$ & & & The median of the pixels in $D_f$ at iteration step $k$.\\
$N_{\text{pix}}$& &  & Number of pixels in each image.\\
$N_{\text{ref}}$& &  & Number of images in the reference cube ($R$).\\
\end{longtable}
\end{center}}
The above table crosses different pages automatically. If you find a way to use the deluxetable directly, please do not use this format since it takes a few minutes to do the conversion.

\newpage
\section{Appendix}
This is your appendix for this chapter.