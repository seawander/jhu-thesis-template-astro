\chapter{Another Chapter in \textit{ApJ} Style}\label{chapter-apj-dup}
%%%%%%%%%%%%%%%%%%%%%%%%%%%%%%%%%%%%%%%%%%%%%%%%%%%%%%%%%%%%%
Note: if you want italic font in your chapter title, use \textbackslash\{\} rather than \textbackslash{it}\{\} to have a better formatting in the Table of Contents.
\paragraph*{Abstract} This is your abstract.

\section{Table example}
Table~\ref{tab:stis-obs} has been presented in \citet{ren17}. It is not presented in \citet{ren18}. I am citing two references just to show you that the Reference section for this table will have two entries.

In the .bib file, you do not have to arrange the entries by their citation order! And you can add more entries---they will \textbf{not} appear on the References section as long as you don't cite them.


\begin{table}[htb!]
\centering
\caption[Summary of {\it HST}/STIS coronagraphic imaging observations as of 2016 December.]{Summary of Public {\it HST}/STIS Observations as of Dec.~2016.}
\label{tab:stis-obs}
\begin{tabular}{ccc}\hline\hline
   & Proposed Aperture Name & Number of Flat-Fielded Files \\ \hline
1  & BAR10                  & 116             \\
2  & {\bf WEDGE A0.6}             & {\bf 228}            \\
3  & {\bf WEDGE A1.0}             & {\bf 493}             \\
4  & WEDGE A1.8             & 86              \\
5  & WEDGE A2.0             & 39              \\
6  & WEDGE A2.5             & 1               \\
7  & WEDGE A2.8             & 5               \\
8  & WEDGE B1.0             & 37              \\
9  & WEDGE B1.8             & 8               \\
10 & WEDGE B2.0             & 9               \\
11 & WEDGE B2.5             & 116             \\
12 & WEDGE B2.8             & 5\\ \hline
\end{tabular}
\end{table}
